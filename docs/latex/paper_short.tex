\documentclass[conference]{IEEEtran}
\IEEEoverridecommandlockouts

\usepackage{cite}
\usepackage{amsmath,amssymb,amsfonts}
\usepackage{algorithmic}
\usepackage{algorithm}
\usepackage{graphicx}
\usepackage{textcomp}
\usepackage{xcolor}
\usepackage{booktabs}
\usepackage{multirow}
\usepackage{array}
\def\BibTeX{{\rm B\kern-.05em{\sc i\kern-.025em b}\kern-.08em
    T\kern-.1667em\lower.7ex\hbox{E}\kern-.125emX}}
\begin{document}

\title{DSCP-Based QoS Framework for MQTT Traffic Prioritization in Software-Defined Networks}

\author{\IEEEauthorblockN{1\textsuperscript{st} Abdurrizqo Arrahman}
\IEEEauthorblockA{\textit{Department of Informatics} \\
\textit{Institut Teknologi Sepuluh Nopember}\\
Surabaya, Indonesia \\
6025251013@student.its.ac.id}
\and
\IEEEauthorblockN{2\textsuperscript{nd} Ahmad Bilal}
\IEEEauthorblockA{\textit{Department of Informatics} \\
\textit{Institut Teknologi Sepuluh Nopember}\\
Surabaya, Indonesia \\
6025251040@student.its.ac.id}
\and
\IEEEauthorblockN{3\textsuperscript{rd} Reinaldi Prasetya}
\IEEEauthorblockA{\textit{Department of Informatics} \\
\textit{Institut Teknologi Sepuluh Nopember}\\
Surabaya, Indonesia \\
6025251043@student.its.ac.id}
}

\maketitle

\begin{abstract}
Internet of Things (IoT) menghasilkan data sensor dalam jumlah besar dengan tingkat urgensi yang berbeda. Data anomaly seperti indikasi kebocoran gas memerlukan pengiriman lebih cepat dibanding data monitoring rutin. MQTT sebagai protokol komunikasi IoT yang ringan tidak menyediakan mekanisme prioritas bawaan, sehingga semua pesan diperlakukan sama saat terjadi congestion. Penelitian ini mengusulkan kerangka kerja QoS berbasis DSCP (Differentiated Services Code Point) untuk MQTT dengan memanfaatkan kemampuan Software-Defined Networking (SDN). Publisher menambahkan nilai DSCP pada header IP menggunakan socket option, kemudian controller SDN menginstal flow rules yang memetakan nilai DSCP ke antrian prioritas menggunakan Hierarchical Token Bucket (HTB) di Open vSwitch. Sistem diuji pada empat skenario dengan dua topologi (hierarchical 13 switch dan dual-redundant 17 switch) untuk mengevaluasi delay, jitter, throughput, dan packet loss pada kondisi congestion 1.8x kapasitas link. Hasil eksperimen menunjukkan traffic anomaly (DSCP 46) mencapai delay 1,110x lebih cepat dan packet loss 0\% dibanding traffic normal, dengan jitter 8.7x lebih rendah untuk predictability yang lebih baik.
\end{abstract}

\begin{IEEEkeywords}
MQTT, SDN, DSCP, QoS, IoT, Traffic Prioritization
\end{IEEEkeywords}

\section{Pendahuluan}
Internet of Things (IoT) menghasilkan data sensor dengan tingkat urgensi yang berbeda. Data anomaly seperti indikasi kebocoran gas membutuhkan pengiriman lebih cepat dibandingkan data pemantauan rutin. MQTT, sebagai protokol komunikasi IoT yang ringan, tidak menyediakan mekanisme prioritas bawaan sehingga seluruh pesan diperlakukan sama. Ketika jaringan mengalami congestion, pesan anomaly berpotensi mengalami keterlambatan.

Software-Defined Networking (SDN) menawarkan pendekatan fleksibel melalui pemisahan control plane dan data plane. Namun, MQTT tidak menyertakan informasi prioritas pada layer jaringan, sehingga switch tidak dapat membedakan pesan kritis dari pesan reguler. Berbagai penelitian sebelumnya telah mencoba menghadirkan prioritas untuk trafik MQTT: PrioMQTT \cite{b1} menggunakan UDP tetapi tidak kompatibel dengan MQTT standar; RT-MQTT \cite{b2} mengintegrasikan MQTT dengan SDN namun kompleks dengan flow rules per-topik; p-MQTT \cite{b3} hanya memprioritaskan di level broker tanpa perlindungan QoS di jaringan. Hingga saat ini, DSCP (Differentiated Services Code Point) standar RFC 2474 belum dieksplorasi untuk prioritas trafik MQTT.

Penelitian ini mengusulkan kerangka kerja QoS berbasis DSCP untuk MQTT dengan SDN. Publisher menambahkan nilai DSCP pada header IP menggunakan socket option, kemudian controller SDN menginstal flow rules yang memetakan DSCP ke antrian prioritas di Open vSwitch. Sistem diuji pada empat skenario dengan dua topologi (hierarchical 13 switch dan dual-redundant 17 switch) untuk mengevaluasi delay, jitter, dan packet loss pada kondisi congestion.

Kontribusi utama penelitian ini: (1) mengusulkan kerangka kerja QoS berbasis DSCP pertama untuk MQTT yang terintegrasi dengan SDN tanpa memodifikasi protokol MQTT; (2) menyajikan arsitektur hybrid sederhana dengan penandaan prioritas di application layer dan pengaturan queue di network layer melalui flow rules per-DSCP yang scalable; (3) melakukan evaluasi komprehensif pada empat skenario dengan dua topologi berbeda.

\section{Related Work}

Penelitian prioritas MQTT dapat dikategorikan menjadi tiga pendekatan utama. Pertama, \textit{broker-level priority} seperti p-MQTT \cite{b3}, Kim \cite{b4}, Tachibana \cite{b5}, dan Akshatha \cite{b6} yang menambahkan priority queue di broker. Pendekatan ini hanya bekerja di application layer sehingga tidak memberikan proteksi QoS di jaringan saat terjadi congestion.

Kedua, \textit{protocol modification} seperti PrioMQTT \cite{b1} yang mengganti TCP dengan UDP dan menambahkan 64-bit priority value, menghasilkan pengurangan RTT 51-79\%. Namun pendekatan ini tidak kompatibel dengan MQTT standar. Testa et al. \cite{b7} mengintegrasikan PrioMQTT dengan TSN untuk bounded delay, tetapi memerlukan infrastruktur TSN yang belum tersedia luas.

Ketiga, \textit{SDN-based integration} seperti RT-MQTT \cite{b2} yang menggunakan middleware RT-NM untuk deep packet inspection dan flow rules per-topik. Meskipun memberikan kontrol end-to-end, arsitektur ini kompleks dan tidak scalable untuk sistem IoT berskala besar.

DSCP (Differentiated Services Code Point) RFC 2474 telah digunakan untuk QoS pada jaringan tradisional. Yaseen et al. \cite{b8} memanfaatkan DSCP pada SDN untuk flow continuity dengan peningkatan 65\%. Namun penelitian tersebut fokus pada trafik generik, bukan prioritas berbasis semantik aplikasi untuk MQTT. Tabel \ref{tab:comparison} merangkum perbandingan pendekatan yang ada.

\begin{table*}[htbp]
\caption{Comparison of MQTT Priority Approaches}
\begin{center}
\begin{tabular}{|l|c|c|c|c|c|c|c|}
\hline
\textbf{Approach} & \textbf{Layer} & \textbf{\textit{MQTT}} & \textbf{\textit{Network}} & \textbf{\textit{DPI}} & \textbf{\textit{Scala-}} & \textbf{\textit{Comple-}} & \textbf{\textit{SDN}} \\
 & & \textbf{\textit{Compat.}} & \textbf{\textit{QoS}} & \textbf{\textit{Req.}} & \textbf{\textit{bility}} & \textbf{\textit{xity}} & \\
\hline
p-MQTT \cite{b3} & Broker & Yes & No & No & Medium & Low & No \\
\hline
Kim \cite{b4} & Broker & Modified & No & No & Medium & Low & No \\
\hline
Tachibana \cite{b5} & Broker & Yes & No & No & Low & Medium & No \\
\hline
Akshatha \cite{b6} & Broker & Yes$^{\mathrm{a}}$ & No & No & Medium & Medium & No \\
\hline
PrioMQTT \cite{b1} & Protocol & No$^{\mathrm{b}}$ & No & No & High & Medium & No \\
\hline
PrioMQTT+TSN \cite{b7} & Proto.+Net. & No$^{\mathrm{b}}$ & Yes & No & Medium & High & No \\
\hline
RT-MQTT \cite{b2} & Middleware & Yes & Yes & Yes & Low & High & Yes \\
\hline
Yaseen \cite{b8} & Network & N/A$^{\mathrm{c}}$ & Yes & No & High & Medium & Yes \\
\hline
\textbf{Proposed} & \textbf{App+Net.} & \textbf{Yes} & \textbf{Yes} & \textbf{No} & \textbf{High} & \textbf{Low} & \textbf{Yes} \\
\hline
\multicolumn{8}{l}{$^{\mathrm{a}}$Uses AMQP. $^{\mathrm{b}}$Uses UDP. $^{\mathrm{c}}$Generic traffic, not MQTT-specific. DPI=Deep Packet Inspection.}
\end{tabular}
\label{tab:comparison}
\end{center}
\end{table*}

Gap utama adalah tidak adanya mekanisme yang menggabungkan DSCP marking berbasis semantik aplikasi dengan SDN untuk MQTT. Penelitian ini mengusulkan kerangka kerja hybrid dengan DSCP marking di publisher dan flow rules per-DSCP yang scalable di SDN.

\section{Methodology}

\subsection{Arsitektur Sistem}
Penelitian ini menggunakan pendekatan eksperimental dengan simulasi jaringan menggunakan Mininet sebagai platform emulasi SDN. Arsitektur sistem terdiri dari tiga layer utama seperti ditunjukkan pada Gambar \ref{fig:architecture}. Pada application layer, MQTT publisher mengirimkan data sensor dan melakukan DSCP tagging pada IP header menggunakan socket option IP\_TOS. Control layer menggunakan Ryu controller yang mengatur flow rules pada switch berdasarkan nilai DSCP. Data layer terdiri dari OpenFlow switches yang menjalankan flow rules dan melakukan queue management menggunakan Hierarchical Token Bucket (HTB) untuk membedakan prioritas traffic.

\begin{figure}[htbp]
\centerline{\includegraphics[width=\columnwidth]{images/Arsitektur-Sistem-DSCP-BasedQoS -MQTT.png}}
\caption{Arsitektur Sistem DSCP-Based QoS untuk MQTT}
\label{fig:architecture}
\end{figure}

Sistem bekerja dengan publisher men-set nilai DSCP pada IP header sebelum paket dikirim. Switch mengarahkan paket ke queue berdasarkan DSCP value, dimana queue prioritas tinggi mendapat alokasi bandwidth lebih besar sehingga paket anomaly dikirim lebih cepat saat terjadi congestion.

\subsection{Topologi Jaringan}
Pengujian dilakukan pada dua topologi dengan kompleksitas berbeda. Topologi pertama (Gambar \ref{fig:topo1}) menggunakan arsitektur hierarchical 3-tier dengan 13 switch (1 core, 3 aggregation, 9 edge) dan 19 host termasuk 1 broker dan 18 publisher. Topologi ini mensimulasikan smart building dengan bandwidth setiap link dibatasi 0.2 Mbps (200 Kbps) untuk menciptakan kondisi congestion sekitar 1.8x.

\begin{figure}[htbp]
\centerline{\includegraphics[width=\columnwidth]{images/first-topology.png}}
\caption{Topologi Hierarchical 3-Tier (13 Switches)}
\label{fig:topo1}
\end{figure}

Topologi kedua (Gambar \ref{fig:topo2}) menggunakan arsitektur dual-redundant dengan 17 switch (2 core, 6 distribution, 9 edge) untuk menguji apakah mekanisme DSCP priority tetap bekerja pada kondisi redundansi penuh dan failover.

\begin{figure}[htbp]
\centerline{\includegraphics[width=\columnwidth]{images/second-topology.png}}
\caption{Topologi Dual-Redundant (17 Switches)}
\label{fig:topo2}
\end{figure}

\subsection{Mekanisme DSCP Priority}
Mekanisme prioritas menggunakan DSCP yang merupakan 6-bit field pada IP header sesuai standar RFC 2474. Penelitian ini mengimplementasikan 5 level prioritas:
\begin{itemize}
\item DSCP 46 (EF): prioritas sangat tinggi untuk data anomaly kritis
\item DSCP 34 (AF41): prioritas tinggi untuk data sensor penting
\item DSCP 26 (AF31): prioritas medium untuk monitoring reguler
\item DSCP 10 (AF11): prioritas rendah untuk background data
\item DSCP 0 (BE): prioritas default untuk data normal
\end{itemize}

Setiap nilai DSCP dipetakan ke queue berbeda di OpenFlow switch dengan alokasi bandwidth proporsional menggunakan HTB scheduler.

\subsection{Algoritma Traffic Prioritization}
Proses prioritisasi dilakukan melalui dua tahap: DSCP tagging di publisher dan traffic differentiation di switch. Pada eksperimen ini fokus pada dua level prioritas utama yaitu DSCP 46 untuk data anomaly dan DSCP 0 untuk data normal, seperti ditunjukkan pada Algorithm \ref{alg:dscp}.

\begin{algorithm}
\caption{DSCP-based Traffic Prioritization}
\label{alg:dscp}
\begin{algorithmic}[1]
\STATE \textbf{Publisher Side:}
\STATE Read sensor data (value)
\IF{sensor = anomaly}
    \STATE dscp\_value $\leftarrow$ 46 \COMMENT{Expedited Forwarding}
\ELSE
    \STATE dscp\_value $\leftarrow$ 0 \COMMENT{Best Effort}
\ENDIF
\STATE sock.setsockopt(IP\_TOS, dscp\_value $\ll$ 2)
\STATE Publish MQTT message with QoS 1
\STATE
\STATE \textbf{Switch Side:}
\STATE Extract DSCP value from IP header
\IF{ip\_dscp = 46}
    \STATE SetQueue(1) + Forward (Bandwidth: 60-80\%)
\ELSIF{ip\_dscp $\in$ \{34, 26, 10\}}
    \STATE SetQueue(2-4) + Forward (Bandwidth: 15-60\%)
\ELSE
    \STATE SetQueue(5) + Forward (Bandwidth: 5-15\%)
\ENDIF
\STATE HTB scheduler processes queues by priority
\end{algorithmic}
\end{algorithm}

\subsection{Implementasi Sistem}
SDN controller diimplementasikan menggunakan Ryu framework 4.34 dengan OpenFlow 1.3. Controller menginstall flow rules ke semua switch berdasarkan nilai DSCP dengan priority value berbeda.

Sistem MQTT menggunakan Mosquitto broker 2.0 pada port 1883. Publisher diimplementasikan dengan Python menggunakan paho-mqtt 1.6. Publisher anomaly menghasilkan nilai sensor dalam range 50-100, sedangkan publisher normal dalam range 20-30. Rate pengiriman 10 msg/s per publisher dengan payload 150 bytes, sehingga 18 publisher dengan overhead protokol menghasilkan total beban sekitar 360 Kbps yang melebihi kapasitas link 200 Kbps (congestion 1.8x).

\subsection{Skenario Pengujian}

Empat skenario dirancang untuk menguji berbagai kondisi jaringan seperti ditunjukkan pada Tabel \ref{tab:skenario}.

\begin{table}[htbp]
\caption{Deskripsi Skenario Pengujian}
\begin{center}
\begin{tabular}{|c|l|c|p{3.5cm}|}
\hline
\textbf{\#} & \textbf{Skenario} & \textbf{Topo} & \textbf{Kondisi} \\
\hline
01 & Baseline & 13 sw & Normal, tanpa gangguan \\
\hline
02 & Lossy & 13 sw & Packet loss pada link \\
\hline
05 & Redundant & 17 sw & STP enabled, dual-core \\
\hline
06 & Failure & 17 sw & Dist. switch mati menit ke-4 \\
\hline
\end{tabular}
\label{tab:skenario}
\end{center}
\end{table}

Setiap skenario dijalankan dengan durasi 10 menit fase pengiriman dan 10 menit fase drain untuk memastikan semua message dalam queue terproses. Pengujian diulang 3 kali untuk memastikan konsistensi statistik.

\subsection{Metrics Evaluasi}
Empat metrics utama digunakan untuk evaluasi QoS:

\textbf{End-to-End Delay}: selisih waktu antara pengiriman dan penerimaan message.
\begin{equation}
\mu_{delay} = \frac{\sum_{i=1}^{n} delay_i}{n}
\end{equation}

\textbf{Jitter}: variasi delay antar message berturutan.
\begin{equation}
jitter_i = |delay_i - delay_{i-1}|
\end{equation}

\textbf{Throughput}: jumlah message per detik yang berhasil diterima.
\begin{equation}
throughput = \frac{total\_messages}{duration}
\end{equation}

\textbf{Packet Loss}: persentase message yang hilang.
\begin{equation}
packet\_loss = \frac{seq_{expected} - seq_{received}}{seq_{expected}} \times 100\%
\end{equation}

Eksperimen dilakukan pada VM Ubuntu 20.04 LTS (4 CPU cores, 8GB RAM) dengan Mininet 2.3.0, Open vSwitch 2.13, dan tools analisis pandas/numpy/matplotlib.

\section{Results}

Bagian ini menyajikan hasil eksperimen dari empat skenario pengujian. Setiap skenario dijalankan 3 kali dan hasilnya dirata-rata untuk memastikan konsistensi.

\subsection{Hasil Per-Skenario}

\textbf{Skenario 01 - Baseline (13 Switches):} Topologi hierarchical tanpa kondisi khusus. Traffic anomaly mendapatkan delay rata-rata 213.92 ms dengan packet loss 0\%, sedangkan traffic normal mengalami delay 237.6 detik dan packet loss 78.18\%.

\textbf{Skenario 02 - Lossy Network (13 Switches):} Kondisi dengan packet loss tambahan pada link. Delay anomaly meningkat menjadi 338.21 ms namun packet loss tetap hanya 0.01\%.

\textbf{Skenario 05 - Dual-Redundant (17 Switches):} Topologi dengan redundansi penuh dan STP enabled. Performa hampir identik dengan baseline (delay 212.76 ms, loss 0\%).

\textbf{Skenario 06 - Distribution Failure (17 Switches):} Switch distribusi dimatikan pada menit ke-4. Kedua traffic mengalami degradasi, namun anomaly tetap lebih baik (41.73\% loss vs 79.23\%).

\subsection{Perbandingan Antar Skenario}

Tabel \ref{tab:scenarios} menyajikan perbandingan komprehensif hasil semua skenario.

\begin{table}[htbp]
\caption{Perbandingan Hasil Semua Skenario (Rata-rata 3 Run)}
\begin{center}
\begin{tabular}{|l|c|c|c|c|}
\hline
\textbf{Skenario} & \multicolumn{2}{c|}{\textbf{Avg Delay (ms)}} & \multicolumn{2}{c|}{\textbf{Packet Loss (\%)}} \\
\cline{2-5}
 & \textbf{Anomaly} & \textbf{Normal} & \textbf{Anomaly} & \textbf{Normal} \\
\hline
01-Baseline & 213.92 & 237,612 & 0.00 & 78.18 \\
\hline
02-Lossy & 338.21 & 237,684 & 0.01 & 78.37 \\
\hline
05-Redundant & 212.76 & 237,891 & 0.00 & 78.25 \\
\hline
06-Failure & 28,848 & 233,840 & 41.73 & 79.23 \\
\hline
\end{tabular}
\label{tab:scenarios}
\end{center}
\end{table}

Pada kondisi operasi normal (skenario 01, 02, 05), traffic anomaly (DSCP 46) konsisten mendapatkan delay 212-338 ms dan packet loss mendekati 0\%, sedangkan traffic normal (DSCP 0) mengalami delay sekitar 237 detik dan packet loss sekitar 78\%. Perbedaan lebih dari 1000x ini membuktikan efektivitas mekanisme DSCP-based QoS.

\subsection{Analisis Jitter dan Throughput}

Tabel \ref{tab:jitter} menyajikan perbandingan jitter dan throughput. Jitter merepresentasikan variasi delay yang penting untuk aplikasi real-time.

\begin{table}[htbp]
\caption{Perbandingan Jitter dan Throughput}
\begin{center}
\begin{tabular}{|l|c|c|c|r|}
\hline
\textbf{Skenario} & \textbf{Jitter A} & \textbf{Jitter N} & \textbf{Ratio} & \textbf{Throughput} \\
\hline
01-Baseline & 60.76 ms & 530.80 ms & 8.7x & 494.87 msg/s \\
\hline
02-Lossy & 103.67 ms & 534.66 ms & 5.2x & 509.97 msg/s \\
\hline
05-Redundant & 60.85 ms & 532.68 ms & 8.8x & 498.20 msg/s \\
\hline
06-Failure & 169.99 ms & 569.65 ms & 3.4x & 391.21 msg/s \\
\hline
\end{tabular}
\label{tab:jitter}
\end{center}
\end{table}

Traffic anomaly memiliki jitter 8.7x lebih rendah dibanding normal pada kondisi baseline, menunjukkan delay yang lebih predictable. Throughput pada skenario normal konsisten sekitar 495-510 msg/s, sementara skenario failure turun menjadi 391 msg/s.

\subsection{Analisis Per-Sensor}

Tabel \ref{tab:persensor} menunjukkan hasil per-sensor pada skenario baseline untuk memverifikasi konsistensi mekanisme prioritas di semua lokasi.

\begin{table}[htbp]
\caption{Hasil Per-Sensor Skenario Baseline}
\begin{center}
\begin{tabular}{|l|c|r|r|r|}
\hline
\textbf{Sensor} & \textbf{Floor} & \textbf{Sent} & \textbf{Recv} & \textbf{Delay} \\
\hline
\multicolumn{5}{|c|}{\textit{Anomaly Traffic (DSCP 46) - Loss: 0\%}} \\
\hline
sensor\_f1r1\_anomaly & 1 & 5,968 & 5,968 & 212.96 ms \\
sensor\_f1r2\_anomaly & 1 & 5,967 & 5,967 & 218.38 ms \\
sensor\_f2r1\_anomaly & 2 & 5,965 & 5,965 & 219.73 ms \\
sensor\_f2r2\_anomaly & 2 & 5,964 & 5,964 & 209.80 ms \\
sensor\_f3r1\_anomaly & 3 & 5,962 & 5,962 & 213.47 ms \\
sensor\_f3r3\_anomaly & 3 & 5,960 & 5,960 & 206.75 ms \\
\hline
\multicolumn{5}{|c|}{\textit{Normal Traffic (DSCP 0) - Loss: ~78\%}} \\
\hline
sensor\_f1r1\_normal & 1 & 5,975 & 1,306 & 237,218 ms \\
sensor\_f2r2\_normal & 2 & 5,971 & 1,315 & 238,677 ms \\
sensor\_f3r3\_normal & 3 & 5,967 & 1,301 & 240,028 ms \\
\hline
\end{tabular}
\label{tab:persensor}
\end{center}
\end{table}

Dari data per-sensor terlihat bahwa:
\begin{itemize}
\item Semua 9 sensor anomaly berhasil mengirim 100\% message tanpa packet loss.
\item Delay antar sensor anomaly sangat konsisten (206-219 ms), menunjukkan fairness dalam antrian prioritas.
\item Sensor normal mengalami packet loss konsisten ~78\% di semua floor dan room.
\item Tidak ada bias terhadap lokasi sensor tertentu dalam topologi.
\end{itemize}

\subsection{Analisis Per-Floor}

Tabel \ref{tab:perfloor} menyajikan agregasi hasil per-lantai untuk melihat apakah ada perbedaan performa berdasarkan posisi dalam topologi.

\begin{table}[htbp]
\caption{Perbandingan Per-Floor pada Skenario Baseline}
\begin{center}
\begin{tabular}{|c|l|r|r|r|}
\hline
\textbf{Floor} & \textbf{Type} & \textbf{Messages} & \textbf{Avg Delay} & \textbf{Loss} \\
\hline
1 & Anomaly & 17,901 & 215.14 ms & 0.00\% \\
1 & Normal & 3,922 & 236,867 ms & 78.10\% \\
\hline
2 & Anomaly & 17,892 & 213.42 ms & 0.00\% \\
2 & Normal & 3,932 & 238,222 ms & 78.07\% \\
\hline
3 & Anomaly & 17,883 & 211.56 ms & 0.00\% \\
3 & Normal & 3,902 & 238,644 ms & 78.17\% \\
\hline
\end{tabular}
\label{tab:perfloor}
\end{center}
\end{table}

Hasil menunjukkan tidak ada perbedaan signifikan antar lantai. Floor 3 yang paling jauh dari core switch memiliki delay anomaly sedikit lebih rendah (211.56 ms) dibanding Floor 1 (215.14 ms), menunjukkan bahwa jarak hop tidak mempengaruhi prioritas secara signifikan.

\subsection{QoS Improvement Summary}

Tabel \ref{tab:improvement} merangkum improvement yang dicapai oleh mekanisme DSCP-based QoS pada skenario baseline.

\begin{table}[htbp]
\caption{Ringkasan QoS Improvement}
\begin{center}
\begin{tabular}{|l|r|r|r|}
\hline
\textbf{Metric} & \textbf{Anomaly} & \textbf{Normal} & \textbf{Improvement} \\
\hline
Avg Delay & 213.92 ms & 237,612 ms & 1,110x faster \\
\hline
Packet Loss & 0.00\% & 78.18\% & 100\% better \\
\hline
Jitter & 60.76 ms & 530.80 ms & 8.7x lower \\
\hline
Msg Delivered & 53,673 & 11,726 & 4.6x more \\
\hline
Std Dev Delay & 43.03 ms & 136,357 ms & 3,169x lower \\
\hline
\end{tabular}
\label{tab:improvement}
\end{center}
\end{table}

Hasil menunjukkan improvement yang sangat signifikan. Traffic anomaly mencapai delay 1,110x lebih cepat dengan 100\% message terkirim, dibandingkan traffic normal yang hanya 21.82\% message berhasil diterima.

\section{Discussion}

Hasil eksperimen dari empat skenario menunjukkan keberhasilan kerangka kerja DSCP-based QoS dalam memprioritaskan traffic MQTT pada kondisi congestion.

\subsection{Efektivitas Mekanisme Prioritas}

Mekanisme HTB queue dengan alokasi bandwidth 60-80\% untuk Queue 1 (DSCP 46) terbukti efektif melindungi traffic prioritas tinggi. Pada kondisi congestion 1.8x kapasitas link, traffic anomaly mencapai:
\begin{itemize}
\item Delay 1,110x lebih cepat (213.92 ms vs 237,612 ms)
\item Packet loss 0\% vs 78.18\% untuk traffic normal
\item Jitter 8.7x lebih rendah (60.76 ms vs 530.80 ms)
\item Std dev delay 3,169x lebih rendah, menunjukkan konsistensi tinggi
\end{itemize}

Perbedaan yang sangat signifikan ini membuktikan bahwa DSCP marking di application layer yang dikombinasikan dengan queue management di network layer dapat memberikan diferensiasi QoS yang efektif tanpa memodifikasi protokol MQTT.

\subsection{Konsistensi dan Fairness}

Analisis per-sensor menunjukkan bahwa mekanisme prioritas bekerja secara konsisten di semua lokasi:
\begin{itemize}
\item Semua 9 sensor anomaly (Floor 1, 2, 3) mendapatkan 0\% packet loss
\item Delay antar sensor sangat konsisten (206-219 ms, variance $<$6\%)
\item Tidak ada bias terhadap lokasi sensor dalam topologi hierarchical
\item Floor 3 yang paling jauh dari core memiliki delay hampir sama dengan Floor 1
\end{itemize}

Hal ini menunjukkan bahwa algoritma HTB scheduler memberikan fairness yang baik di antara traffic dengan prioritas yang sama.

\subsection{Skalabilitas pada Topologi Kompleks}

Perbandingan antara topologi 13 switches dan 17 switches menunjukkan skalabilitas yang baik:
\begin{itemize}
\item Delay: 213.92 ms (13 sw) vs 212.76 ms (17 sw) - perbedaan $<$1\%
\item Packet loss: 0\% pada kedua topologi
\item Throughput: 494.87 msg/s vs 498.20 msg/s
\end{itemize}

Penambahan 4 switch dan aktivasi STP tidak mempengaruhi performa mekanisme prioritas secara signifikan. Hal ini penting untuk deployment pada jaringan enterprise yang memerlukan redundansi.

\subsection{Ketahanan Terhadap Kondisi Abnormal}

\textbf{Lossy Network:} Pada kondisi lossy, delay meningkat 58\% (dari 213.92 ms menjadi 338.21 ms) karena retransmisi TCP. Namun, reliabilitas tetap terjaga dengan packet loss hanya 0.01\%.

\textbf{Network Failure:} Pada kondisi failure (switch distribusi mati pada menit ke-4), kedua traffic mengalami degradasi. Namun, traffic anomaly tetap mendapatkan perlakuan lebih baik:
\begin{itemize}
\item Packet loss anomaly: 41.73\% vs normal: 79.23\%
\item Delay anomaly: 28.8 detik vs normal: 233.8 detik
\end{itemize}

Tingginya packet loss anomaly pada skenario failure disebabkan oleh message yang sedang dalam perjalanan saat failure terjadi dan waktu STP reconvergence (30-45 detik).

\subsection{Perbandingan dengan Pendekatan Lain}

Tabel \ref{tab:approachcomp} membandingkan hasil penelitian ini dengan pendekatan sebelumnya.

\begin{table}[htbp]
\caption{Perbandingan dengan Pendekatan Lain}
\begin{center}
\begin{tabular}{|l|c|c|c|}
\hline
\textbf{Approach} & \textbf{Delay Improv.} & \textbf{Network QoS} & \textbf{Complexity} \\
\hline
p-MQTT \cite{b3} & Moderate & No & Low \\
\hline
PrioMQTT \cite{b1} & 51-79\% RTT & No & Medium \\
\hline
RT-MQTT \cite{b2} & High & Yes & High \\
\hline
\textbf{Proposed} & \textbf{1,110x} & \textbf{Yes} & \textbf{Low} \\
\hline
\end{tabular}
\label{tab:approachcomp}
\end{center}
\end{table}

Dibandingkan dengan p-MQTT \cite{b3} yang hanya bekerja di broker, pendekatan ini memberikan proteksi QoS di network layer. Dibandingkan dengan RT-MQTT \cite{b2} yang memerlukan middleware dan deep packet inspection, DSCP marking lebih sederhana karena hanya memerlukan socket option di publisher dan flow rules per-DSCP (bukan per-topik) yang lebih scalable.

\subsection{Limitasi dan Future Work}

Beberapa limitasi penelitian ini:
\begin{itemize}
\item Pengujian dilakukan pada lingkungan emulasi (Mininet), bukan jaringan fisik
\item Hanya menguji 2 level prioritas (DSCP 46 dan DSCP 0)
\item Belum menguji dengan jumlah publisher yang sangat besar ($>$100)
\end{itemize}

Future work dapat mencakup: evaluasi pada hardware switch fisik, implementasi dynamic priority adjustment, dan integrasi dengan TSN untuk deterministic latency.

\section{Conclusion}

Penelitian ini mengusulkan dan memvalidasi kerangka kerja QoS berbasis DSCP untuk prioritas traffic MQTT pada jaringan Software-Defined Networking. Kerangka kerja ini mengkombinasikan DSCP marking di application layer dengan queue management di network layer tanpa memerlukan modifikasi pada protokol MQTT standar.

Hasil eksperimen komprehensif pada empat skenario dengan total 12 run menunjukkan efektivitas pendekatan ini:

\begin{itemize}
\item \textbf{Efektivitas}: Traffic anomaly (DSCP 46) mencapai delay 1,110x lebih cepat (213.92 ms vs 237,612 ms) dan packet loss 0\% dibanding traffic normal pada kondisi congestion 1.8x.

\item \textbf{Konsistensi}: Semua 9 sensor anomaly di 3 lantai berbeda mendapatkan perlakuan yang sama dengan delay variance $<$6\%, menunjukkan fairness yang baik.

\item \textbf{Skalabilitas}: Performa hampir identik pada topologi 13 dan 17 switches (perbedaan $<$1\%), menunjukkan skalabilitas untuk deployment enterprise.

\item \textbf{Predictability}: Jitter 8.7x lebih rendah (60.76 ms vs 530.80 ms), penting untuk aplikasi yang memerlukan timing konsisten.

\item \textbf{Ketahanan}: Pada kondisi failure, prioritas tetap dipertahankan dengan traffic anomaly mendapatkan packet loss lebih rendah (41.73\% vs 79.23\%).
\end{itemize}

Kontribusi utama penelitian ini meliputi: (1) kerangka kerja QoS berbasis DSCP pertama untuk MQTT yang terintegrasi dengan SDN, (2) arsitektur hybrid sederhana dengan DSCP marking di publisher dan flow rules per-DSCP yang scalable, (3) validasi eksperimental komprehensif pada empat skenario dengan dua topologi berbeda.

Dibandingkan dengan pendekatan sebelumnya, kerangka kerja ini memiliki keunggulan: kompatibel dengan MQTT standar (tidak seperti PrioMQTT), memberikan proteksi QoS di network layer (tidak seperti p-MQTT), dan tidak memerlukan deep packet inspection (tidak seperti RT-MQTT).

Future work dapat mencakup: (a) evaluasi pada hardware switch fisik untuk validasi pada kondisi produksi, (b) implementasi dynamic priority adjustment berdasarkan kondisi jaringan real-time, (c) integrasi dengan Time-Sensitive Networking (TSN) untuk aplikasi industrial yang memerlukan deterministic latency, dan (d) pengujian dengan jumlah publisher yang lebih besar ($>$100) untuk validasi skalabilitas ekstrem.

\begin{thebibliography}{00}
\bibitem{b1} G. Patti, L. L. Bello, and L. Leonardi, ``PrioMQTT: An MQTT Extension Supporting Prioritized Communications,'' IEEE Trans. Ind. Informat., 2024.
\bibitem{b2} M. Shahri, S. Movahedi, and Z. Movahedi, ``RT-MQTT: A Real-Time MQTT Framework for SDN-based Networks,'' in Proc. IEEE Conf. Network Softwarization (NetSoft), 2022.
\bibitem{b3} S. Kim, H. S. Kim, and C. Y. Lee, ``p-MQTT: A Priority-based MQTT for Real-Time Sensor Data Transmission,'' in Proc. IEEE Int. Conf. Ubiquitous Inf. Technol. Appl. (CUTE), 2018.
\bibitem{b4} Y. Kim, ``Priority-based Message Delivery Mechanism for IoT Services,'' in Proc. Int. Conf. Inform. Netw. (ICOIN), 2017.
\bibitem{b5} K. Tachibana, K. Shimizu, and T. Hasegawa, ``Prioritization Control Mechanism for IoT Communications,'' in Proc. IEEE Symp. Comput. Commun. (ISCC), 2016.
\bibitem{b6} Akshatha et al., ``Priority-based Message Queuing using RabbitMQ for IoT Applications,'' in Proc. Int. Conf. Adv. Comput. Commun. Syst. (ICACCS), 2024.
\bibitem{b7} A. Testa, G. Patti, and L. Leonardi, ``Integration of PrioMQTT with Time-Sensitive Networking,'' IEEE Access, 2024.
\bibitem{b8} M. Yaseen et al., ``DSCP-based Traffic Flow Continuity Control in SDN Networks,'' J. Netw. Comput. Appl., 2024.
\end{thebibliography}

\end{document}
